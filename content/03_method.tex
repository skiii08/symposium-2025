\section{提案手法}

\subsection{モデルの基本構造}

提案手法 GLR(Group-wise Linear Regression with Regularization)は, 
評価予測を次の線形モデルで表す:

\[
\hat{y} = b + \sum_{g=1}^{17} w_g^\top x_g
\]

ここで, 予測評価値を $\hat{y}$, バイアス項を $b$, 
特徴グループ(全17グループ)を $g$, 
重みベクトルを $w_g$, 特徴ベクトルを $x_g$ とする.
本研究の核心は, 17 グループを意味的に 4 系列に集約することで, 
評価要因を明確化する点にある:

\[
\hat{y} = b + \mathrm{User} + \mathrm{Movie} + \mathrm{Interaction} + \mathrm{Review}
\]

各系列の役割は以下の通りである:

・User 系列(686 次元):レビュアーの俳優・監督嗜好の埋め込み, 
ジャンル別評価傾向, レビュー統計から抽出した感情特性を含む.

・Movie 系列(1202 次元):映画の俳優・監督・キーワード埋め込み, 
タグ, ジャンル情報を格納する.

・Interaction 系列(80 次元):User 嗜好と Movie 属性の類似度や 
要素積により相性を表現する.

・Review 系列(22 次元):個別レビューのトピック別感情スコアを捉える.

この 4 系列分解により, 線形構造の性質上, 
各系列の寄与は $w_g^\top x_g$ の和として数学的に一意に決定される.
そのため, 「どの要因がどの程度影響したか」を安定的に解釈できる.

\subsection{レビュアー個性を引き出す正則化}

Movie 系列は次元数(1202 次元)が大きく, 放置すると寄与が過度に優勢になる.
これを抑え, User 系列に個性を適切に反映させるため, 3 種の正則化を導入した.

(1) Movie Suppression  
Movie 系列の寄与を抑え, User/Movie のバランスを保つ.
これにより, 実際に見られる「人による評価差」を線形モデルで再現できる.

(2) Inter-group Correlation  
User と Movie の重み相関を下げ, 両系列が独立した情報を保持するよう制御する.

(3) Intra-group Variance  
高次元で特定次元に重みが集中することを防ぎ, 多面的な嗜好表現を確保する.

これらの正則化により, User 系列は「個人の判断基準」, Movie 系列は「作品の特徴」を明確に分離でき, 
この構造が本研究における個性抽出の基盤となる.

\subsection{データセット}

IMDb データセットから得た 735 ユーザー・1247 作品・36003 レビューを用い, 
70/12/18% で Train/Val/Test に分割した. 評価値はユーザー内 Z-score 正規化により 
個人差を補正した. 特徴量には, TMDb から取得したメタデータ, 
GPT-4 によるトピック・感情分類, FastText(300 次元)による埋め込みを用いた.
