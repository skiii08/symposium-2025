\section{性能と評価要因の抽出}

\subsection{予測性能}

GLR は Test MAE が 0.656 となり, Simple Linear(0.865)比で 24\% 改善した.
同一レビュアー内での User 寄与分散(0.581)は, 同一映画に対する Movie 寄与分散(0.285)の約 2 倍であり,
評価が作品の一般的特徴よりもレビュアー固有の判断基準に強く依存することを示している.

\subsection{評価要因の定量化とクラスタ分析}

User 系列から抽出した 6 軸要因(actor, director, genre, content,
interaction, background)に基づき K-means($k=6$)で評価傾向を分類した.
得られた 6 パターンは次の通りである:

・Actor-Focused(87 名):俳優寄与が突出し,出演者に強く反応する  

・Director-Focused(130 名):監督の作家性や演出傾向を主要判断基準とする  

・Genre-Focused(135 名):作品のジャンル構造に敏感で,類型的特徴を重視する  

・Content-Focused(113 名):内容・テーマ・物語的要素を中心に評価する  

・Multi-Signal(76 名):複数軸を組み合わせて総合的に判断する  

・Balanced(194 名):6 軸が均衡しており特定要素への偏りが小さい

図1 は各クラスタの 6 軸寄与を可視化したもので,Actor/Director の優位性,
Genre/Content の対照性,Balanced の均衡構造など,特徴的な差異が確認できる.

\begin{figure}[htbp]
\centering
\includegraphics[width=0.95\linewidth]{figures/heatmap_profiles_gray.pdf}
\caption{6 種類の評価パターンにおける 6 軸寄与のヒートマップ}
\label{fig:heatmap}
\end{figure}
