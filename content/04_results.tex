\section{性能と評価要因の抽出}

\subsection{予測性能}

GLR は Test MAE が 0.656 となり, Simple Linear(0.865)比で 24\% 改善した.
同一レビュアー内での User 寄与分散(0.581)は, 同一映画に対する Movie 寄与分散(0.285)の約 2 倍であり,
評価が作品の一般的特徴よりもレビュアー固有の判断基準に強く依存することを示している.

\subsection{評価要因の定量化とクラスタ分析}

User 系列から抽出した 6 軸要因(actor, director, genre, content, interaction, background)を用いて,
レビュアーごとの評価傾向を定量化した. 6 軸の平均ベクトルに対して K-means($k=6$)を適用した結果,
以下の 6 種の評価パターンが得られた:

\textbf{Actor-Focused}(87 名), 
\textbf{Director-Focused}(130 名), 
\textbf{Genre-Focused}(135 名), 
\textbf{Content-Focused}(113 名), 
\textbf{Multi-Signal}(76 名), 
\textbf{Balanced}(194 名).

図 1 はそれぞれの 6 軸寄与を可視化したものであり,
Actor-Focused と Director-Focused の突出した違い,
Genre-Focused と Content-Focused の構造的対照性,
Balanced の均衡傾向が確認できる.

\begin{figure}[htbp]
\centering
\includegraphics[width=0.95\linewidth]{figures/heatmap_profiles_gray.pdf}
\caption{6 種類の評価パターンにおける 6 軸寄与のヒートマップ}
\label{fig:heatmap}
\end{figure}
